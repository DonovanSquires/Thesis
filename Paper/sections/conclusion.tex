The work of many students in the Neurobiology Engineering Laboratory at Western Michigan University made the completion of the Data Acquisition and Stimulation System (DASS) possible.  Utilizing the documentation and knowledge gained from previous groups~\cite{ArmstrongSD,McCaskeyPackaralJohnVandeusen07,CarusoDaiekJones,StahlMSEE,BatzerCorsiCrampton} and the concurrent efforts of fellow graduate student, Mr.~Kyle Batzer~\cite{BatzerMSEE}, a custom multi-channel data acquisition and stimulation system has been developed for use in electrophysiology experiments at the Neurobiology Engineering Laboratory.  Validation of the system by performing a standard electrophysiology experiment was, again, aided by the experience of students' previous design projects~\cite{StahlMSEE,EllingerMSEE}.

Designing hardware for a mixed digital and analog system can be a challenge.  Being able to refer to previous work reduces design time and risk.  Thorough documentation in this thesis also reduces risk and aides in the subsequent use of the system.  A PCB can be an expensive investment (see appendix~\ref{sec:bom}), and the utilization of headers and $0\unit{\Omega}$ resistors in the design of the Electrophysiology Interface board allows the few layout mistakes to be rectified, as shown in appendix~\ref{sec:errata}.

Application of the DASS by future students in electrophysiology experiments will require knowledge of electrical engineering.  The hardware, even though it does contain some protection circuitry, can be damaged by misapplication of power or incorrect connection of inputs and outputs.  The documentation in this thesis will make understanding the DASS attainable for an electrical engineering student who devotes a month or two to part-time study of the system.  The documentation can also be used for a future project that would implement a system, described in section~\ref{sec:scaling}, with enough channels to fully exploit an Microelectrode Array (MEA) with 60 electrodes.  Such a project may be too ambitious for a senior design group or graduate student to complete alone in one year, but a senior design group or graduate student that has experience working with the DASS on electrophysiology experiments before beginning such a project could realize a working 60 channel data acquisition and stimulation system.

Personally, I broadened my knowledge of electronic circuit design in creating the Electrophysiology Interface board for the DASS, gained experience collaborating with a talented firmware and software designer, explored another field by learning enough neurophysiology to perform and design hardware for electrophysiology experiments, and gained confidence in my circuit and PCB design skills by having a PCB manufactured that was used successfully in an electrophysiology experiment.

