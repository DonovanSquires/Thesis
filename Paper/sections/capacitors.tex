Capacitors are populated on the RTSC board and the Electrophysiology Interface board for the purpose of maintaining quality power supply voltages in the presence of high switching currents and radiated and conducted electromagnetic interference (EMI).  Capacitors connected to the power and ground planes perform three functions: decoupling which provides a source of DC power near components that meets demand for short duration power surges and keeping RF energy generated by components from propagating into the power supply network, bypassing which provides a low impedance path to ground for high frequency noise that may interfere with components, and bulk capacitance that is used to maintain a constant DC voltage for situations that cause a large $\mathrm{d}i/\mathrm{d}t$, such as all digital signals switching simultaneously~\cite{Montrose1999}.

Ideal capacitors with large capacitance values yield a low impedance at high frequencies and store more energy than low capacitance values, but the electrolytic and tantalum constructions that allow large capacitance values in small volumes have comparatively large equivalent series resistance (ESR) and equivalent series inductance (ESL) that limit their usefulness at higher frequencies and for large $\mathrm{d}i/\mathrm{d}t$.  Thus low ESR, ESL, and value ceramic capacitors are often used in parallel with electrolytic or tantalum capacitors to provide the required capacitance for decoupling and bypassing~\cite{Montrose1999}.  The capacitors are also placed as close as possible to the component that needs bypassing and decoupling to limit the added inductance in the length of PCB trace between the capacitor and the component.  Because of resonance effects with parallel capacitors, their values should be separated by at least two orders of magnitude (i.e. the value of the electrolytic capacitor should be at least 100 times greater than the value of the ceramic capacitor)~\cite{Montrose1999}.  Power and ground planes, themselves being two parallel plates, also act as capacitors with the advantage of having effectively no ESR and very low ESL and may provide sufficient capacitance, with closely spaced planes for slow switching frequency and slew rate circuits, but the planes should be spaced by less than $0.1\unit{in}$ and vias limit the effective capacitance of the parallel planes~\cite{Montrose1999}.

On the Electrophysiology Interface board, a capacitor that performs decoupling and bypass functions is populated for each power supply pin of a component and each power pin of a connector, and the capacitor is placed as close as possible to the pin.  One $0.1\unit{\mu F}$ ceramic capacitor is used, if no recommendation is given by the part's data sheet.  The AD5678 data sheet recommends a $10\unit{\mu F}$ capacitor in parallel with a low ESR $0.1\unit{\mu F}$ capacitor~\cite{AD5678ds}.  If a surface mount ceramic capacitor is available with the recommended capacitance, a ceramic capacitor is used even if the data sheet suggests using a tantalum or electrolytic capacitor.  One large value electrolytic capacitor is available for each of the +5VD, +VA, and $\unit{-VA}$ power supplies to provide bulk capacitance.
