To create a durable, reliable, and professional-looking implementation of the Data-Acquisition and Stimulation System hardware, it is desirable to have the Electrophysiology Interface board design implemented on a printed circuit board (PCB).  To create a PCB, a layout of the circuit design is developed and converted to an industry standard file format that can be submitted to a PCB manufacturer who constructs the PCB.  The complexity of the Electrophysiology Interface board circuit design necessitated a four-layer PCB construction, with internal power and ground plane layers that reduce noise emissions and susceptibility while simplifying signal routing and component placement on the top and bottom layers.

Developing the layout of the PCB requires an electronic design automation (EDA) software suite that is capable of working with a complex design implemented on a four-layer PCB and exporting the PCB design to the industry standard Gerber file format.  KiCad is an open-source EDA suite that has been in development since 2006; provides schematic capture, PCB layout design, and Gerber file export at no cost; and is available for all popular operating systems.  Other EDA suites were evaluated: freeware software suites are often tied to a specific manufacturer with no included Gerber file export capabilities, full-featured software often has prohibitively expensive licensing fees or evaluation versions suitable for only small designs, and the other popular open-source suite, gEDA, does not have an official Windows binary distribution. 

Advanced Circuits is the company that manufactured the Electrophysiology Interface PCB based on Gerber files created using KiCad.  The manufacturer was chosen based on its location in the USA, Gerber file acceptance, and attractive student pricing that produces a four-layer board with silkscreen and solder mask for \$66 with no minimum board quantity requirement.

