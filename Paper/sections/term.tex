\paragraph{Analog to Digital Converter (ADC)} A device for converting an analog voltage to a digital code.  The ADC on the Electrophysiology Interface is an Analog Devices AD7606.

\paragraph{Complex Programmable Logic Device (CPLD)} Non-volatile programmable logic device that is used to allow limited outputs from the FPGA to control many more digital input lines required by multiple Preamp boards connected to the Electrophysiology Interface.  The CPLD on the Electrophysiology Interface may be a Xilinx\textsuperscript{\textregistered} XC9536XL or a XC9572XL (the devices are pin compatible).

\paragraph[Data Acquisition and Stimulation System \(DASS\)]{Data Acquisition and Stimulation System (DASS)\footnote{This item is co-authored with Kyle Batzer,~\cite{BatzerMSEE}.}\addtocounter{footnote}{-1}\addtocounter{Hfootnote}{-1} }  System intended for electrophysiology experiments described in this thesis and corresponding thesis~\cite{BatzerMSEE}.  The DASS includes all hardware, software, and firmware.

\paragraph[Data Acquisition and Simulation Control Center \(DASCC\)]{Data Acquisition and Simulation Control Center (DASCC)\footnotemark}  PC application, described in~\cite{BatzerMSEE}, for controlling and transferring data to and from the RTSC.

\paragraph{Differential Output Amplifier} A custom designed operational amplifier configuration on the Electrophysiology Interface that provides gain and offset to an input from the DAC and provides two output signals where one signal is the inverse of the other signal, with respect to ground (0V).

\paragraph{Digital to Analog Converter (DAC)} A device for converting a digital code to an analog voltage.  The DAC on the Electrophysiology Interface is an Analog Devices AD5678.

\paragraph[Electrophysiology Interface]{Electrophysiology Interface\footnote{This item is co-authored with Kyle Batzer,~\cite{BatzerMSEE}.}\addtocounter{footnote}{-1}\addtocounter{Hfootnote}{-1}}  Subsystem that provides the RTSC with an interface to biological systems.  It consists of a custom PCB described in this thesis.

\paragraph{Field Programmable Gate Array (FPGA)} Volatile programmable logic device that performs the majority of the real time control functions on the RTSC.  The FPGA on the RTSC is a Xilinx\textsuperscript{\textregistered} Spartan-3 XC3S500E.

\paragraph[Real Time System Controller \(RTSC\)]{Real Time System Controller (RTSC)\footnotemark\addtocounter{footnote}{-1}\addtocounter{Hfootnote}{-1}}  Subsystem that implements the real-time functions of the system.  It consists of a Digilent\textsuperscript{\textregistered} Nexys\textsuperscript{TM} 2 development board~\cite{DigilentNexys2rm,DigilentNexys2sch}, with custom firmware described in~\cite{BatzerMSEE}.

\paragraph[Preamp]{Preamp\footnotemark}  Low noise instrumentation amplifier with stimulation dc bias addition for MEA experiments.  Designed by Mr. John Stahl as described in~\cite{StahlMSEE}.  

\paragraph{Printed Circuit Board (PCB)} A fiberglass board with thin layers of copper that are etched and drilled allowing electronic components to be mounted and connected in a compact, professional-looking package for an electronic circuit.




