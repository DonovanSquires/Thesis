The next step in designing the PCB in the KiCad EDA suite is to assign layout footprints to the components included in the schematic with the program CvPcb, which reads the netlist file created by Eeschema.  Through-hole footprints are utilized for connectors, protection diodes, and electrolytic capacitors.  Surface mount footprints are utilized for all other components to limit footprint size and lead length inductance~\cite{Montrose1999}.

Solder mount through-hole footprints with $0.1\unit{in}$ ($2.54\unit{mm}$) spacing are used for connectors with the exception of the Hirose FX2 and PCI-Express connectors.  Resistors, ceramic capacitors, and LEDs use standard 0805 package outline footprints as a compromise between choosing a package that minimizes the board space requirement and one that is large enough to be easy to solder by hand.  Small-Outline (SO) variations of integrated circuits are preferred when available for the component, and Quad-Flat Package (QFP) variations are used for the 64-pin count components.  Test points are implemented with a square pad of plated copper which minimizes required area on the board while allowing a test probe to connect to the signal and provides an acceptable surface for soldering a wire if rework is needed.

Custom footprints not included in the KiCad suite are used for several components and are included in the $lib$ directory under the main project directory.  The footprints for the Preamp PCI-Express connector and the CPLD are based on libraries from~\cite{osheclib}, and the footprint for the 100-pin Hirose FX2 connector is based on a library from~\cite{CodingAdvFX2}.  The footprint for the 16-pin terminal block for connecting to the recording electrodes was modified with a larger hole size to enable two 8-pin terminal blocks to be used side-by-side~\cite{TETermBlockDraw}.  Guide holes were modified to be plated to be compatible with the design constraints of the manufacturer~\cite{AdvCir66}.  All component footprints were compared to the recommended PCB footprints in the components' respective data sheets to ensure compatibility.

CvPcb saves the footprint information to the netlist file created by Eeschema.  The name for the footprints of each component can be seen in the $Package$ column on the bill of materials in Table~\ref{tab:bom} in Appendix~\ref{sec:bom}.