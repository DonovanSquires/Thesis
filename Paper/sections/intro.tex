An Integrated Electrophysiology Data Acquisition and Stimulation System to support electrophysiology research has been developed by building on previous work at WMU~\cite{ArmstrongSD,CarusoDaiekJones,StahlMSEE,BatzerCorsiCrampton,EllingerMSEE}, studying commercial systems~\cite{MCSystems}, and reviewing the research literature~\cite{Potter2005,Jimbo2003,Blum2007}.  The developed prototype provides a real-time platform for measurement and stimulation of biological electrical activity and a PC application for controlling the real-time platform and visualizing cellular activity.  The system can accommodate up to eight measurement channels and four stimulation channels, and the design can be expanded for up to 64 channels to support future research at the Western Michigan University (WMU) Neurobiology Engineering Laboratory.
 
The developed prototype is viable for a wide array of electrophysiology experiments, completely fulfilling the instrumentation needs of~\cite{Olivo,KuehJellies,Kladt2010} and partially fulfilling the requirements for~\cite{Marom02,DeMarse04,Potter2005}.  In particular, a standard electrophysiology experiment was performed on earthworm giant axon action potentials to validate system functionality.  The prototype is also intended for studying software and hardware principles required for performing research using cells cultured on a Microelectrode Array (MEA), e.g.~\cite{MCSystems}.  A cell culture protocol has been developed~\cite{EllingerMSEE} and previous work on such a system~\cite{ArmstrongSD,CarusoDaiekJones,BatzerCorsiCrampton}, including low noise amplification, has been completed~\cite{StahlMSEE}.  Initial analytical algorithms have also been developed~\cite{EllingerMSEE}.
